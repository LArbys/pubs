
{\pubs} is kept at a public github repository. One can access the respotiroy
through a web browser as well
\href{https://github.com/drinkingkazu/pubs}{\color{blue}\bf https://github.com/drinkingkazu/pubs}.
In this section we cover how to obtain {\pubs} and configure to use the tools.
We also cover how to clean up your local database server (as far as {\pubs} concerns).

\subsection{Obtain And Configure {\pubs}}
\label{prep:pubs:conf}
First of all, make a git account. Then let the author know about your account so he
can add you as a collaborator, and you will have a write permission. This makes your
life much easier.

If you have a write permission, you can checkout as shown below:
\begin{lstlisting}
  > git clone git@github.com:drinkingkazu/pubs pubs
  > cd pubs
  > git checkout devel
\end{lstlisting}
If you have only a read permission, try the following instead:
\begin{lstlisting}
  > git clone https://github.com/drinkingkazu/pubs pubs
  > cd pubs
  > git checkout devel
\end{lstlisting}
Above operations are needed per making a clean checkout from git repository.
You do not need to do them if you are to work on previously checked out repository.

Now, per login, you have to source some configuration scripts as shown below.
\begin{lstlisting}
  > source config/setup.sh
  > source config/personal_conf.sh
\end{lstlisting}
The first script important shell environmnet variables to use {\pubs}. In 
particular, after running the script, you should see a shell environment
variable {\ttfamily \$PUB\_TOP\_DIR} is set:
\begin{lstlisting}
  > echo $PUB_TOP_DIR
\end{lstlisting}
which should be pointing at the top directory of {\pubs}. You can execute
this script from anywhere (i.e. it does not need to be run from {\pubs} top
directory and it will figure out the correct path).

The second script should be really made per user (so you can make your own).
This shell script sets some environment variables that are used to connect
{\psql} server. Take a look at it. How each environment variable is used 
should be trivial from their name. If you have a specific user name, password,
server DNS, database name etc. configured differently than default, make
your own configuration script!

\subsection{Setting Up {\psql} Server For {\pubs}}
\label{prep:pubs:sql}
The {\psql} server interaction model in {\pubs} is to use a pre-defined set of functions 
to make data queries rather than allow users to come up with a random (and probably wild)
bare query strings. These functions are defined in {\sql} scripts:
\begin{itemize}
  \item db\_scripts/master\_creation.sql
  \item db\_scripts/procdb\_functions.sql
\end{itemize}
Uploading these functions to your server is a one-time operation, and you can do:
\begin{lstlisting}
  > psql -h localhost -d procdb -f db_scripts/master_creation.sql
  > psql -h localhost -d procdb -f db_scripts/procdb_functions.sql
  > psql -h localhost -d procdb -f db_scripts/initialize_procdb.sql
\end{lstlisting}

Now, someday you might be in a situation such that something went wrong on the server
and you want to reset all. Well, it is a beauty of working on localhost of your machine:
you can go wild and reset {\it everything}. Here's how you can do this:
\begin{lstlisting}
  > dropdb procdb
  > createdb procdb
  > psql -h localhost -d procdb -f db_scripts/master_creation.sql
  > psql -h localhost -d procdb -f db_scripts/procdb_functions.sql
  > psql -h localhost -d procdb -f db_scripts/initialize_procdb.sql
\end{lstlisting}

Try this solution if something weird happened to your {\procdb} database.
But just remember one thing: this operation removes {\it everything} including 
all project tables. Be responsible ;)

\subsection{Running {\pubs}}
\label{prep:pubs:daemon}
Make sure you have read and followed Sec.\ref{prep:pubs:conf} and \ref{prep:pubs:sql}.
Now you should be ready to execute a test script to see things are set up correctly.
Try:
\begin{lstlisting}
  > cd $PUB_TOP_DIR
  > python dstream/ds_daemon.py
\end{lstlisting}
This should start-up a process that continues indefinitely making {\stdout} messages
every second. The script you are running is actually one of core tools in {\pubs} that
can execute all projects that you will develop (or developed by others) within {\pubs}.
For now, this part is complete. We will come back to in the later chapters.
