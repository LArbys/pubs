
This chapter describes a prerequisite for using {\pubs}. In particular we
cover installation of database related external packages as it uses {\psql}.
Find more about {\psql} on their home web page \cite{PSQLHome}.

\section{Prerequisites}
In this section, we go over prequisites to use {\pubs}. None of those require any
particularly intensive computation power, and they can be all installed on a 
personal laptop.

\subsection{Operating System And Software}
As {\pubs} does not require a compilation, it is expected to work on any operating system (OS).
That being said, some functionality of underlying {\python} does depend on the kernel. 
The author has not tested on Windows family OS, and he highly doubt it would work on it. 
There should be no problem on Unix and Linux family OS. 

We use {\git} for a software repository and version control. As mentioned, {\pubs} use
{\php}, {\psql}, and {\python}. A summary of minimaum version for these software is given
below. These are oldest version that the author happened to find on his machines, so it
is possible that things work on even older version of kernel/software. Please tell the 
author or update the document if you find such case.

\begin{itemize}
\item Supported Operating Systems
    \begin{itemize}
    \item Mac OSX 10.5.8 (Darwin 9.8) or later with Xcode 3.1.4 or later
    \item Linux 2.6.0 or later with \gpp 4.1 or later
    \end{itemize}
\item Version Control Tool
    \begin{itemize}
    \item {\git} $\ldots$ 1.7.1 or later
    \end{itemize}
\item Base Software Tools
    \begin{itemize}
    \item {\php} $\ldots$ v2.1 or later
    \item {\psql}$\ldots$  v9.1 or later with {\ttfamily contrib} for {\hstore}
    \item {\python} $\ldots$ v2.7 or later (but no later than 3.0)
    \item {\psycopg} $\ldots$ v2.5 or later
    \end{itemize}
\end{itemize}
Both recent OSX and Linux distributions have {\python} 2.7 and {\git}. So we skip
those items. The next section discuss how to install some rather uncommon parts
including {\psql}, {\hstore}, and {\psycopg}.

Both {\psql} and {\python} has a very good documentation and FAQ resource on the web.
You may find Ref.\cite{PythonDoc,PSQLDoc} useful.

\section{Installation of {\psql} And Extensions ({\psycopg} and {\hstore})}

{\psycopg} is a {\python} interface to {\psql} server (Ref.\cite{Psycopg2}). 
As far as the author can tell, this is not a part of {\python} 2.7 standard module 
libraries. So you have to install it as an external product.

{\hstore} is an external data representation introduced in {\psql} (Ref.\cite{HSTORE}).
It is a part of {\psql} {\ttfamily contrib} which is not included in a default set
of libraries for compilation. We use this as well.

{\pubs} itself works without a locally running {\psql} server. 
However, if you are to work on code development within {\pubs} framework, it is handy
to work on your own local {\psql} server because most likely you will be ``messing''
with the database and some test/dummy projects. So I spend some space here to
share how to install {\psql} and {\psycopg} on your machine. 

\subsection{Words on Installation Method}
If you use Linux, you might say ``Well I would just use {\aptget} or {\yum}!'', or 
``I have a {\fink}/{\brew}/{\macport}!'' if you use OSX family.
There's nothing wrong to use {\aptget} or {\yum} in principle, but the author has
seen some OS specific default configuration imposed when one installs using those
tools that is different from what one gets by compiling from the source. 
To be on the same page among readers with different OS, the author suggests to 
simply compile from source. And, by the way, in the author's biased mind, {\fink} 
and {\brew} are even out of questions. Sorry.

But a compilation from source obviously requires a compiler. You can obtain this 
using your package distributor if you use Linux. Or {\xcode} if you use OSX.

\subsection{Installing {\psql}}
Go visit 
\href{http://www.postgresql.org/ftp/source/}{{\color{blue}\bf http://www.postgresql.org/ftp/source/}}
to obtaib the source code of {\psql}. Currently the author sees 9.3.5 as the latest production 
version and is therefore recommended. Click the source link and download, and untar it wherever
you would like to install locally. In case you do not know how to untar it, if your file extension
is {\ttfamily .tar.gz}, try:
\begin{lstlisting}
  > tar -xvfz postgresql-9.3.5.tar.gz
\end{lstlisting}
If it has simply {\ttfamily .tar} extension, try:
\begin{lstlisting}
  > tar -xvf postgresql-9.3.5.tar
\end{lstlisting}
which is often the case on OSX because it automatically un-zip it.
Now in the extracted directory (in the example above it should be ``postgresql-9.3.5''), you
find {\ttfamily INSTALL} text file that tells you how to install. You can follow and type
the followings.
\begin{lstlisting}
  > cd postgresql-9.3.5
  > ./configure
  > make
  > sudo make install
\end{lstlisting}
Now from the next steps, {\ttfamily INSTALL} probably tells you to make a user {\ttfamily postgres}
as the master of the database server. Assuming you are installing this locally on your own 
laptop/desktop and planning to be the master of the server forever, I suggest you do not make
{\ttfamily postgres} and just use your account. That means to do this:
\begin{lstlisting}
  > sudo mkdir /usr/local/pgsql/data
  > chown $USER /usr/local/pgsql/data
  > /usr/local/pgsql/bin/initdb -D /usr/local/pgsql/data
  > /usr/local/pgsql/bin/postgres -D /usr/local/pgsql/data >logfile 2>&1 &
\end{lstlisting}
If above command gives you a trouble, and if you give up solving on your own, feel free to
contact the author to solve the mess as you followed his suggestion so far!

Now if above commands are successful, add the following lines in your {\ttfamily \$HOME/.bashrc}
or {\ttfamily \$HOME/.bash\_profile}:
\begin{lstlisting}
  export PATH=/usr/local/pgsql/bin:$PATH
  export LD_LIBRARY_PATH=/usr/local/pgsql/lib:$LD_LIBRARY_PATH
\end{lstlisting}
If you use OSX, you also need this:
\begin{lstlisting}
  export DYLD_LIBRARY_PATH=/usr/local/pgsql/lib:$DYLD_LIBRARY_PATH
\end{lstlisting}
To enforce these shell environment variables, simply close your current shell and open
a new one to work on.

\subsection{Creating Database}
Now that your server is running, let's create a database to work on. Among {\pubs} developers,
we use a database called {\procdb} to develop our codes. You can make your own name, but sharing
a database name probably makes it easier to communicate. So the author recommend to use this.
Anyhow, you can create a database easily like this:
\begin{lstlisting}
  > createdb procdb
\end{lstlisting}
If you get an error message from your terminal complaining ``no such file or command'', then that
is because you have not yet set your shell environment variables as discussed in the previous
paragraph. If you get an error message regarding a permission, that is probably because you made
{\ttfamily postgres} account and/or you have not changed a permission of 
{\ttfamily /usr/local/pgsql/data} directory, where both are discussed in the previous paragraph.
Contact the author if you have a hard time solving these issues.

Now you should be able to log-in to your local psql server:
\begin{lstlisting}
  > psql -h localhost -d procdb 
  psql (9.3.5)
  Type ``Help'' for help.
  procdb=# 
\end{lstlisting}
which is your {\psql} interpreter in {\procdb} database.

\subsection{Installing {\hstore} Extensions}

First, let's compile {\hstore}. To do this, you need your shell environment variable set 
correctly as mentioned earlier. After you follow those steps, go back to 
your {\psql} source directory, then try:
\begin{lstlisting}
  > cd contrib/hstore
  > make
  > sudo make install
\end{lstlisting}
This should compile {\hstore} for you. Again, if you have a problem, contact the author.
Now this only made a library of {\hstore} available to your server, and it does not mean 
{\hstore} became available in your database. You have to enable {\hstore} in your database 
explicitly. Try this:
\begin{lstlisting}
  > psql -h localhost -d procdb
  procdb=# CREATE EXTENSION HSTORE;
\end{lstlisting}
Do not forget ``;'' in the end. This installs {\hstore} data type and all related functions
to your {\procdb} database.

\subsection{Installing {\psycopg}}

You can download {\psycopg} source code from 
\href{https://pypi.python.org/pypi/psycopg2}{\color{blue}\bf https://pypi.python.org/pypi/psycopg2}.
{\psycopg2} uses {\ttfamily distutils}, which is a standard {\python} package build tool.
To compile, simply type:
\begin{lstlisting}
  > python setup.py build
  > sudo python setup.py install
\end{lstlisting}
Note, again, this requires you to have set the shell environment variables like {\ttfamily \$PATH}
and {\ttfamily \$LD\_LIBRARY\_PATH}. Nevertheless this is all you need to do. To test whether this
worked or not, try:
\begin{lstlisting}
  > python
  >>> import psycopg2
\end{lstlisting}
If this does not throw an error message, you're good. One further test you can do:
\begin{lstlisting}
  > import os
  >>> host = 'localhost'
  >>> user = os.environ['USER']
  >>> db   = 'procdb'
  >>> pass = ''
  >>> psycopg2.connect(host=host,user=user,database=db,password=pass)
\end{lstlisting}
If this does not throw an error message, you're beautiful. This means your local {\psycopg}
library is correctly linked against your {\psql}.

